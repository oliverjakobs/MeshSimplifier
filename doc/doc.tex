\documentclass[a4paper,12pt]{book}

\usepackage[a4paper, inner=1.7cm, outer=2.7cm, top=1.5cm, bottom=2cm, bindingoffset=1.2cm]{geometry}
\usepackage[german]{babel}
\usepackage{microtype}
\usepackage{fancyhdr}


\pagestyle{fancy}
\fancyfoot{}
\fancyfoot[C]{\thepage}

\begin{document}

\chapter*{Dokumentation zur Modulaufgabe:}

\section*{Thema und Motivation}

\section*{Bibliotheken}

Um für mein Projekt ein Fenster mit OpenGL-Context erstellen zu können und Zugriff auf Maus und Tastatur zu haben, verwende ich die Bibliothek \textbf{GLFW}. Ich habe mich für GLFW entschieden weil ich damit die meiste Erfahrung hab und gut damit klar komme.

Um mir den Umgang mit OpenGL zu vereinfachen, verwende ich das von mir selbst entwickeltes Framework \textbf{Ignis}. Teil dieses Frameworks ist auch die OpenGL-Loading-Library \textbf{GLAD}.

DearImGui

glm

\section*{Designentscheidungen}

Kein Export zu .obj o.ä. - trivial, und ich würde im normalfall in ein Engine-Spezifisches Format exportieren

Edges und Neighbor multimap - edges nur zum Speichern der Kantenkosten und zum Bestimmen der nächsten Kante zum entfernen

Vertices nur Positionen - macht Algorithmus einfacher und übersichtlicher

Flat-Shading - ermöglicht Vertices mit nur Positionen, außerdem hebt das die Faces hervor

Vertices werden nicht geupdated/entfernt (nur Indices) - würde ich erst beim exportieren machen

\section*{Bedienung}

\section*{Build}


\end{document}