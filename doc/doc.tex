\documentclass[a4paper,12pt]{book}

\usepackage[a4paper, inner=1.7cm, outer=2.7cm, top=1.5cm, bottom=2cm, bindingoffset=1.2cm]{geometry}
\usepackage[german]{babel}
\usepackage{microtype}
\usepackage{fancyhdr}
\usepackage{hyperref}
\usepackage[pdftex]{graphicx}

\pagestyle{fancy}
\fancyfoot{}
\fancyfoot[C]{\thepage}

\newcommand{\myparagraph}[1]{\paragraph*{#1}\mbox{}\\}

\begin{document}

\chapter*{Dokumentation zur Projektaufgabe für das Modul Tool- und Pluginprogrammierung im Sommersemester 2022}

\section*{Thema und Motivation}

Als Thema für meine Modulaufgabe habe ich mich für \textbf{Entwicklung einer Mesh Decimation-Anwendung} entischieden, da ich in meinem nächsten privaten Projekt mit 3D-Meshes arbeiten will und ich dann potenziell Verwendung für einen Mesh Decimation Algorithmus finde. 
\\
\\
Während meiner Recherche bin ich auf ein Paper mit dem Titel '\emph{Surface Simplification Using Quadric Error Metrics}' von Michael Garland und Paul S. Heckbert gestoßen und habe mich dafür entschieden den in diesem Paper vorgestellten Algorithmus (zumindestens teilweise) zu implementieren. Da in dem Paper von Mesh oder Surface Simplification anstelle von Mesh Decimation gesprochen wird, werde ich das Verfahren auch Simplification nennen. 
\\
\\
Da ich am Anfang bisschen überfordert war und nicht wirklich wusste wo ich anfangen soll, habe ich nach anderen Implementionen gesucht, durch die ich den Algorithmus besser verstehen kann. Dabei habe ich ein GitHub Repository (\url{https://github.com/Meirshomron/MeshSimplification}) gefunden, das mir vor allem bei der Umsetzenung der Fehlerberechnung geholfen hat.

\newpage

\section*{Bibliotheken}

In diesem Projekt verwende ich vier bzw. fünf Bibliotheken, die es mir ermöglichen eine grafischen Anwendung zu entwickeln, in der der Algorithmus an einem Modell ausgeführt werden kann.

\begin{itemize}
	\item \textbf{GLFW (\url{https://www.glfw.org/}):}
	Für das Erstellen des OpenGL-Kontextes und den Zugriff auf Maus und Tastatur verwende ich GLFW. 
	Ich habe mich dafür entschieden, da ich damit die meiste Erfahrung habe und gut damit klar komme.
	\item \textbf{Ignis (\url{https://github.com/oliverjakobs/Ignis}):}
	Ignis ist ein von mir selbst entwickeltes Framework, das mir für meine Projekte den Umgang mit OpenGL vereinfachen soll. 
	Mit Ignis kann ich einfacher OpenGL Objekte wie Shader oder VertexArrays erstellen.
	\item \textbf{GLAD (\url{https://glad.dav1d.de/}):} 
	Glad ist eine OpenGL-Loading-Library und wird von Ignis eingebunden.
	\item \textbf{Dear ImGui (\url{https://github.com/ocornut/imgui}):}
	Dear ImGui ist eine eine Bibliothek zum Erstellen von Grafischen Benutzeroberflächen basierend auf dem \emph{Immediate Mode GUI Paradigma}.
	Durch Dear ImGui kann ich meiner Anwendung eine grafische Benutzeroberfläche geben, und somit die Bedienung vereinfachen.
	\item \textbf{glm (\url{https://github.com/g-truc/glm}):}
	Glm ist eine Mathe-Bibliothek basierend auf den \emph{OpenGL Shading Language (GLSL) Spezifikationen}. 
	Ich benutze glm, damit ich mir keine Gedanken machen muss, ob alle mathematischen Operationen auch wirklich richtig funktionieren. 
\end{itemize}

\section*{Designentscheidungen}

Meine Designentscheidungen unterteile ich in zwei Abschnitte. Im ersten Abschnitt geht es um allgemeine Entscheidungen, bei denen es um den Umfang des Projektes oder die Benutzerschnittstelle geht. Im zweiten Abschnitt geht es dann um Entscheidungen, die die eigentliche Implementation des Algorithmus betreffen.

\subsection*{Allgemeine Entscheidungen}

\myparagraph{Umfang des Projektes}
Das Programm soll ein 3D-Model mit dem Format \texttt{.obj} laden und anzeigen können. Dann soll über ein GUI ausgewählt werden können auf wie viele Faces das Mesh reduziert werden soll. Durch einen Button soll dann der Algorithmus mit den vorher festegelegtem Ziel ausgeführt werden. Außerdem soll es einen Button geben, der das Mesh auf seinen ursprünglichen Zustand zurücksetzt.

Des Weiteren habe ich mich gegen den Export in ein Dateiformat entschieden, da es von der Komlexität eher trivial ist und ich im Normalfall in ein Engine-Spezifisches Format exportieren würde.

\myparagraph{Rendering}
Für das Renderering habe ich für Flat-Shading entschieden. Dadurch werden die Faces eines Meshes hervorgehoben und die Veränderungen durch die Mesh-Simplification werden offensichtlicher.
Außerdem können die für das Flat-Shading benötigten Face-Normals im Fragment-Shader berechnet werden, wodurch es möglich ist, dass die Vertices, was das Rendern betrifft, nur aus Positionen bestehen. 

\myparagraph{Ein- und Ausgabe des Algorithmus}
Wenn ein Vertex also einfach nur ein \texttt{vec3} ist, wird das Laden des Meshes einfacher (und schneller) und ich habe für die Eingabe in den Simplification-Algorithmus einen Vektor für die Vertices (also die Positionen) und einen Vektor für die Indices, die die Faces bilden. Diese beiden Vektoren gibt der Algorithmus dann auch wieder (verändert) aus.

\subsection*{Entscheidungen zur Algorithmus Implementation}

\myparagraph{Darstellung der Vertices}
Für den von mir gewählten Algorithmus muss einem Vertex eine Position und eine Fehler-Matrix zugeordnet werden können.
Das realisiere ich durch zwei Vektoren (einen für die Positionen und einen für die Fehler-Matrizen), indem ich den Vertex durch einen Index darstelle.
Um jetzt an die Position eines Vertex \emph{v} zu gelangen, muss nur der \emph{v}-te Eintrag in den Vector der Positionen ausgelesen werden. Für die Fehler-Matrix eines Vertex funktioniert das analog.

Der Grund warum ich mich für diese Darstellung der Daten entschieden hab ist, weil ich dadruch den geringsten Aufwand bei der Ein- und Ausgabe in bzw. aus dem Algorithmus habe. 
Ich will nicht die Positionen in Vertices mit Fehler-Matrizen umrechnen, oder die Fehler-Matrizen außerhalb des Algorithmus mit rumschleppen müssen.

\myparagraph{Paare}
In dem oben erwähnten Paper wird das Mesh vereinfacht, indem aus einer Liste aus allen möglichen Paaren das günstigste Paar ausgewählt wird. Dabei ist es nicht nötig, dass zwischen den beiden Verices eines Paares auch eine Kante existiert, solange die Distanz zwischen den Vertices einen festgelegten Threshold nicht überschreitet. 
Um den Umfang von meinem Projekt nicht zu groß werden zu lassen, habe ich mich entschieden diesen Threshold auf Null zu setzen und nur Paare zuzulassen, zwischen dessen Vertices auch eine Kante existiert.

Beim Erstellen dieser Paare verwende ich ein \texttt{std::unordered\textunderscore set} um Duplikate zu verhindern.
Dabei sind zwei Paare $a$ und $b$ identisch, wenn:
$$(a.first=b.first \land a.second=b.second) \lor (a.second=b.first \land a.first=b.second)$$

Die Paare werden (mit \texttt{std::make\textunderscore heap}) in einem Min Heap verwaltet, wobei die Sortierung anhand des Fehlers des Paares erfolgt.

\myparagraph{Ablauf des Algorithmus}
In dem Paper wird der Algorithmus in fünf Schritten zusammengefasst.

\myparagraph{Veränderung der Vertices}
Für jedes Paar, das vereint wird, wird einer der beiden Vertices verändert.
Für den anderen Vertex werden lediglich die Faces und Paare, von denen er ein Teil ist angepasst.
Der eigentliche Vertex wird aber nicht entfernt, es gibt nur keinen Index mehr der auf ihn verweist.
Das wirkliche entfernen der überflüssigen Vertices würde ich erst beim exportieren durchführen.

\newpage
\section*{Bedienung}

Im ersten Abschnitt des GUIs (A) werden die Anzahl der Vertices und der Faces angezeigt. 
Außerdem können hier die verfügbaren Meshes ausgewählt und geladen werden (B). Zu Verfügung stehen ein einfacher Würfel (8 Vertices, 12 Faces), ein Affenkopf (2012 Vertices, 3936 Faces) und die Marsienne Base (13235 Vertices, 27894 Faces) aus Übung03.

Der zweite Abschnitt (C) ist die Bedienung des Simplifiers. Hier kann mit einem Slider die gewünschte Anzahl an Faces ausgewählt werden (der Slider geht von 0 bis zur Anzahl der Faces der aktuellen Version des Meshes). Mit dem Button \texttt{Simplify} wird das Mesh so lange vereinfacht, bis die gewünschte Anzahl an Faces erreicht wurde. Mit dem Button \texttt{Reset} wird die ursprüngliche Version des Meshes wieder hergestellt.

Im letzten Abschnitt (D) befinden sich zwei Render-Optionen mit denen der Wireframe-Modus ein- und ausgeschaltet bzw. das Backface Culling aktiviert und deaktiviert werden kann.

\begin{figure}[h]
	\centering
	\includegraphics[width=\linewidth]{gui.png}
\end{figure}

Im Hauptbereich der Anwenung wird die aktuelle Version des Meshes gerendert. Mit Hilfe der linken Maustaste kann die Kamera auf einem Arcball um das Mesh bewegt werden.

Mit \texttt{Esc} kann das Programm beendet werden.

\section*{Build}

Zur Project Generation benutze ich Premake (\url{https://premake.github.io/}). Ich habe die entsprechenden Scripts meiner Abgabe beigefügt.
Falls also Probleme mit der VisualStudio Solution autreten sollten,
kann mit dem folgenden Befehl das Projekt neu generiert werden und so die Probleme hoffentlich gelöst werden:
\[\texttt{.\textbackslash premake\textbackslash premake5.exe [action]}\]
Für mein Projekt habe ich \texttt{vs2019} als \texttt{action} verwendet. Andere Möglichkeiten sind hier \url{https://premake.github.io/docs/Using-Premake} aufgelistet.


\end{document}